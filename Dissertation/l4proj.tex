\documentclass{l4proj}
\usepackage{url}
\usepackage{indentfirst}
\usepackage{graphicx}
\begin{document}
\title{Level 4 Project Report - A Large Scale Pedometer Application}
\author{David Mcinnes\\ \\0901288m@student.gla.ac.uk}
\date{29th March 2014}
\maketitle

\begin{abstract}
An project with a number of aspects - a pedometer application, a website and associated backend software.
\end{abstract}

\educationalconsent
\tableofcontents
%==============================================================================
\pagenumbering{arabic}

\chapter{Introduction}

The aim of this project was to produce and implement a large scale pedometer app using phones and the Internet in order to provide a large scale infrastructure for pedometer usage. This is interesting because fitness and wearable technology is going to be an ever increasing area of importance in technology and Human Computer Interface. The near ubiquity of Gyroscopes and Accelerometers in devices such as modern smartphones and tablets mean that this is more feasible than ever before. Additionally, some high end smart phones as of 2013 have included 'Coprocessors' in order to provide a specialised processor to facilitate the tracking of such user activity. The iPhone 5S has this ability, as well as some Android smartphones. 

The premise of the application was such that users would prefer to not have to carry around separate devices in order to track their walking activity - it would be good if one device that eveyone carries around regardless was used. In today's world, most people carry around mobile phones, with most of these being smart-phones in 2014. This means that the modern mobile phone with its near ubiquty and in-built Accelerometer and Gyroscope is the perfect candidate.


\section{Background}

A current area of intense research and interest in the field of Human Computer Interaction is wearable technology. Items that purportedly increase the users likelihood of carrying out exercise are increasing in popularity. Many applications, features and products have been introduced in the past couple of years such as x that are designed to gather information about the users physical habits. 

Wearable technology and technology that is designed to increase the users health levels is going to be an increasingly important part of technology as time goes on - it is possible that software such as this could be of a massive benefit to health in general.

With this in mind, there are different ways to provide a niche for a new Pedometer application. It would be interesting if there was a communal way of tracking all of the users of an application communally, with communal information about usage statistics amongst all of the applications users. This could be useful in a whole variety of contexts.

In the next year or so, the prevalence of such devices will increase massively in everyday usage.

\section{Project Outline}

The goal of this project was to create a pedometer for Android based phones that runs in the background over the user's day. When they leave the house in the morning, they turn the application on, and they simply leave it running whilst they go about their daily business. When running in the background, the application tracks the number of steps the user is making and uploads them to a central server hosted by the department. 

%==============================================================================

\chapter{Literature Review}

Before undertaking the project it was necessary to undertake a review of recent literature from similar topics.

There are a number of products on the market that act as a pedometer for users to keep track of their steps. An example of this is the 'Activity Log' application in Nintendo's 3DS Handheld video game system. Whilst the system is in the users pocket whilst they walk around, it tracks the number of steps.

There were a variety of research papers that informed my approach to designing and implementing the evaluation.

UbiFit garden was an early smartohone application that aimed to provide a stimulus to increase user activity of the mobile phones. Given the era that this application was develeoped in, the application requires additional hardware bolted onto the back of the phone. Given modern conveniences, many people would consider this to be an unacceptable drawback to the application, but for a time when smartphones did not have included accelerometers and gyroscopes to be able to internally handle such applications.

%==============================================================================

\chapter{Design}

Before development began, it was necessary to decide what technology and OS the app would be implemented in. Given the popularity of iOS and Android in modern mobile phones, I decided to choose between these two. Development in iOS uses the Objective C language, which is predominantly used by Apple. I didn’t have any experience developing using the Objective C. Additionally, iOS development requires the use of the XCode IDE, which is only available on Mac OS X.  While this would be possible in my circumstance, it would be more logistically challenging given that it was likely that development would take place on a number of different machines. However, with iOS there are a far smaller number of possible devices that i would have to design around and test for.

Android development is cross-platform on Linux, OS X and WIndows, meaning I could use any computer for development. This would make development logistically easier. Additionally, Android development uses the Java programming language, which I have a lot of experience in, having done it in many courses and having used it as the implementation language for my Level 3 Project. There are a lot of tutorials and guides available for both iOS and Android development.

Given the available choices, I decided to develop the system for Android OS due to my experience with Java, making it less likely that development would reach a snag or be slower due to my inexperience with Objective C. In future, the app could be ported to iOS if usage of he app was shown to be popular.

When considering the overall design of my application, it was necessary to consider many different factors. A major one of these was the prevalence of many differing factors.

Before development began, it was necessary to decide what technology and OS the app would be implemented in. Given the popularity of iOS and Android in modern mobile phones, I decided to choose between these two. Development in iOS uses the Objective-C language, which is predominantly used by Apple. I didn’t have any experience developing using the Objective-C. Additionally, iOS development requires the use of the XCode IDE, which is only available on Mac OS X.  While this would be possible in my circumstance, it would be more logistically challenging given that it was likely that development would take place on a number of different machines. However, with iOS there are a far smaller number of possible devices that i would have to design around and test for.

Android development is cross-platform on Linux, OS X and WIndows, meaning I could use any computer for development. This would make development logistically easier. Additionally, Android development uses the Java programming language, which I have a lot of experience in, having done it in many courses and having used it as the implementation language for my Level 3 Project. There are a lot of tutorials and guides available for both iOS and Android development. One of the major obstacles involed with choosing Android development over iOS is that there are a much greater number of devices that use Android at any given time versus devices that use iOS. Whilst iOS devices are generally all within the high end range of current technology, Android devices can vary wildly in CPU speed, internal memory, and screen resolution. This means making a possibly complex decision about which type of devices to target development for.

Given the available choices, I decided to develop the system for Android OS due to my experience with Java, making it less likely that development would reach a snag or be slower due to my inexperience with Objective C. In future, the app could be ported to iOS if usage of the app was shown to be popular.

There were a variety of different factors to be considered for the website aspect of the project.

%==============================================================================

\chapter{Implementation}

In order to achieve the goals of the project, there was a lot new programming languages to add to my skillset and all of that. For example, I had no knowledge of PHP but had to learn in order to implement the back end for my project.

\section{Android Application}

One of the major concerns throughout implementation of this project was to get the application to such a state whereby battery life is not endangered by constant usage of the application as it is designed to be. Modern smartphones generally have poor battery life given their high processing power, high-resolution screens and regular network operations. These operations can happen regularly, and if not managed in an effienct manner can cause a real negative impact towards the battery life of a device. If an application is causing the user's battery life to be at an unacceptable level reguarly, they would rather delete the application than continue using it.

The first stage of implementation involved getting to grips with Android development, which I had little experience in. This involved reading tutorials on the Android developers site, which has a good introduction to Adroid development that teaches me the basics, such as learning about views and activities, as well as show to travel between and move data between views using intents.

The application consists of 7 classes. In Android development, files describing the layout, look and feel of the User Interface are written using Extensible Markup Language (XML), which represents different views and UI components in a hirarchical structure. The functionality of the app itself is implemented using the Java programming language. Such a technique seperates the look and feel of the app from the actual functionality - separation of concerns. Errors in the UI code doesn't cause errors within the Java code.

The pedometer code that I was offered was written by Mattias Rost, and had been in use by the department for several other projects before my own. This code was not very well documented, however, and I was unsure how to instantiate the code, make it run and begin to track the users steps.  After some correspondance with the original author of the code, it became apparent how I should edit the code in order to make it suit my circumstances.

\section{Back end and Database}

The database necessary to support the functionality of the application and website by storing user information.

The back end is written using the scripting language PHP. Scripts hosted on the server provide access to information from the website and the application. There are a group of scripts assocated with serving the application, and a group of scripts that are associated with providing content for the website. 

//LIST OF ALL THE SCRIPTS GO HERE!!!!!1111

The scripts on the Application are hit by using HTTP Post requests. This HTTP is sent using Apache's HTTP library provided with the Android SDK in the Application. 

The scripts used for the Website are also hit using HTTP Post requests. This HTTP is sent to the server in the form of JQuery requests. JQuery is an API for the Javascript programming language that adds features to the language and greatly improves the interactivity between websites and users, and database between website.

\section{Website}

The website aspect of the project. TO BE COMPLETED!!!!!!!!11111

In order to provide graphing for the website, I used a Javascript library called AwesomeChartJS which rapidly simplified the act of making appropriate bar charts for my website.

%==============================================================================

\chapter{Testing and Evaluation}

\section{Testing}

One of the challenging aspects of this project was actually testing the application. Pedometers are hard to test in that you need to walk around for large periods of time in a realistic environment.

\section{Evaluation}

Upon completion of development of the app and website, it was necessary to carry out a user evaluation to test how well I had achieved what I set out to do and what could be implemented in future from the eyes of the users of the application.

When thinking about participants to use in the evaluation, it was necessary to consider a wide range of users. There were a number of instant limitations to the group of users that could take part in the user evaluation. The application developed only works on 

One issue with the evaluation as carried out was that the participants were not being compensated for taking part in the user evaluation, and a worry was that the users would not bother to use the application for any length of time. This could also be seen as a good thing, for, if example, the users did not feel they had to sugar-coat their feelings because they were being recompensated.
 
Due to the privacy concerns raised by the application, it was necessary to gather user consent from all of my participants. This was carried out in a two stage process - 1) When the user had agreed to be a part of the evaluation, they were made to read and sign a form that explained to them that the application and website would be tracking information about the number of steps they made, and location data about where they used it. It was also explained to them the reasons for carrying out the evaluation, my name and contact address for if they wanted to contact me about any concerns or questions that they had. Finally, it was explained to them that they could leave the evaluation at anytime if they felt uncomfortable, and that there should be no embarrassment for doing so.

During the user evaluation, there was regular bug updates and features added to the application in response to user feedback.

The pedometer application developed was then evaluated.

The conclusion was then written.

%==============================================================================

\chapter{Conclusion}

In conlusion, I fucking loved this project and I think it's hella cool. Please give me an A.

I developed an app and evaluated it.

I developed a website and evaluated it.

I developed a back-end and tested it.

Then I wrote this fucking dissertation.

Then I was finished, thank fuck.

\section{Future Work}

In future, there are a number of features that I feel should be implemented, as backed up by my user study. Whilst the application is programmed to be large scale, if the large scale became properly large then the current set up might have a very hard tome.

Given that a large percentage of smartphone users use the iOS mobile Operating System, there are a large number of prospective users who are simply unable to use the application whilst there is no iOS version of the application. If the app became popular or it was decided that more users were needed, an iOS version could help facilitate this. One thing that it would be necessary to guarantee is that the level of functionality could be guaranteed across any of the platforms that the app runs across - users must be able to have the same accounts between all of the different platforms, and be able to log into the website and see the shared data regardless of what platform it originated on.

Once this was implemented, it would be good at looking at extending the functionality of the apps themselves - the evaluation suggested that users think that these are fairly bare-bones and lacking in features. The first step in executing this would be to move the functionality of the website to also be available on the applications. Having access to richer data w

*LIST ALL OF THE FEATURES HERE GAWD*

%==============================================================================

\chapter{Appendix}

\section{Glassary of common terms and abbreviations}

Android - An open-source Mobile Operating System with development lead by Google. Currently the most popular Mobile OS, with XX percent of the market as of March 2014.

HTTP - Hyper Text Transfer Protocol

POST Request -

GET Request -

Apache Server -

APACHE License -

Accelerometer -

Gyroscope -

LIST STUFF HERE!!!!

\end{document}